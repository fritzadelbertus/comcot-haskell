\documentclass{beamer}
\usepackage[utf8]{inputenc}
\usepackage[T1]{fontenc}

\title{Project Report}
\subtitle{\textit{Week 1}}
\date[]{2024/2025}
\author[Fritz]{Fritz Adelbertus Sitindaon}

\usetheme{Warsaw}
\setbeamertemplate{footline}[frame number]
% ===== Setup Font =====
\usepackage[sfdefault,lf]{carlito}
\usepackage[T1]{fontenc}
\renewcommand*\oldstylenums[1]{\carlitoOsF #1}

% ==== Import Math Packages =====
\usepackage{amsmath, amssymb, amsthm}
\usepackage{mathtools}

% ==== There required =====

\usepackage[most]{tcolorbox}
\def\wallpaper{wallpaper.jpg}

\def\R{\mathbb{R}}
\def\P{\mathbb{P}}
\def\N{\mathbb{N}}
\def\O{\mathbb{O}}


\def\nX{\mathcal{X}}
\def\nY{\mathcal{Y}}
\def\nT{\mathcal{T}}
\def\nU{\mathcal{U}}
\def\nB{\mathcal{B}}
\def\nS{\mathcal{S}}
\def\nP{\mathcal{P}}
\def\nA{\mathcal{A}}
\def\nF{\mathcal{F}}
\newcommand{\comp}[1]{\overline{#1}} 

\DeclarePairedDelimiter\abs{\lvert}{\rvert}
\DeclarePairedDelimiter\floor{\lfloor}{\rfloor}
\DeclarePairedDelimiter\cic{[ }{] }
\DeclarePairedDelimiter\oic{( }{] }
\DeclarePairedDelimiter\cio{[ }{) }
\DeclarePairedDelimiter\oio{( }{) }
\DeclarePairedDelimiter\set{\{ }{\} }
\DeclarePairedDelimiter\brk{(}{)}
\DeclarePairedDelimiter\seq{\langle}{\rangle}

\begin{document}

\begin{frame}
\titlepage
\end{frame}



% \input{pembagian/materi_sebelumnya}
\begin{frame}{What is COMCOT?}
    \begin{itemize}
        \item COMCOT adalah alat pemodelan 
        numerik yang dikembangkan terutama 
        untuk mensimulasikan propagasi gelombang 
        tsunami, run-up, dan inundasi. 
        \item Fitur pada comcot meliputi penggunaan 
        persamaan air dangkal, menerapkan sistem multi-grid,
        dan mengintegrasi kondisi sumber patahan dan tanah longsor.
    \end{itemize}
\end{frame}

\begin{frame}{Proses Pembentukan dan Perambatan Tsunami}
    \begin{center}
        \includegraphics[scale=0.3]{figure/how_tsunami_works.png}
    \end{center}
\end{frame}

\begin{frame}{How the code was modularize}
    \begin{enumerate}
        \item Terjadinya patahan atau tanah longsor di bawah laut. $deform.f90,\,landslide.f90$.
        \item Pembentukan awal gelombang tsunami. $wavemaker.f90,\,dispersion.f90$.
        \item Perambatan gelombang tsunami ke daerah pesisir. $mass.f90,\,moment.f90$.
    \end{enumerate}
    \begin{itemize}
        \item Penyettingan solusi numerik $type\_module.f90,\,all\_grids.f90,\,boundaries.f90$.
        \item Penyettingan program $comcot.f90,\,initialization.f90,\,hotstart.f90,\,output.f90$.
    \end{itemize}
\end{frame}

\begin{frame}{File Structure and Size}
    \begin{itemize}
        \item $type\_module.f90$: 180 lines
        \item $initialization.f90$: 2021 lines
        \item $all\_grids.f90$: 927 lines
        \item $boundaries.f90$: 439 lines
        \item $comcot.f90$: 526 lines
        \item $wavemaker.f90$: 362 lines
        \item $deform.f90$: 675 lines
        \item $landslide.f90$: 299 lines
        \item $mass.f90$: 205 lines
        \item $moment.f90$: 939 lines
        \item $dispersion.f90$: 1647 lines
        \item $hotstart.f90$: 138 lines
        \item $output.f90$: 1010 lines
    \end{itemize}
\end{frame}

\begin{frame}{File Structure and Size}
    \begin{itemize}
        \item $comcot.ctl$: file untuk mengisi input simulasi
        \item $fault\_multi.ctl$: file input tambahan jika sumber patahan lebih dari satu
        \item $landslide.ctl$: file input jika tsunami berasal dari tanah longsor
    \end{itemize}
\end{frame}

\begin{frame}{Porting initialization.f90 ke initialization.hs}
    \begin{itemize}
        \item initialization berisi setup awal program.
        \item initialization meliputi pembacaan input dan konfigurasi dari $.ctl$, 
        pembacaan data-data awal, pensettingan grid dan fungsi-fungsi helper
    \end{itemize}
\end{frame}

\begin{frame}{Porting initialization.f90 ke initialization.hs}
    $READ\_CONFIG$
    \begin{itemize}
        \item $READ\_CONFIG$ membaca semua parameter dari $comcot.ctl$.
        \item $READ\_CONFIG$ juga menyetel grid.
        \item $READ\_CONFIG$ berisi $\pm$ 150 line untuk membaca $comcot.ctl$.
    \end{itemize}
    \includegraphics[scale=0.5]{figure/read_config_2.png}
        
\end{frame}

\begin{frame}{Porting initialization.f90 ke initialization.hs}
    $READ\_CONFIG$
    \includegraphics[scale=0.5]{figure/read_config_1.png}
\end{frame}

\begin{frame}{Porting initialization.f90 ke initialization.hs}
    $READ\_MULTIFAULT\_DATA$
    \begin{itemize}
        \item $READ\_MULTIFAULT\_DATA$ membaca semua parameter dari $fault\_multi.ctl$.
        \item $READ\_MULTIFAULT\_DATA$ berisi $\pm$ 50 line untuk membaca $fault\_multi.ctl$.
    \end{itemize}
    \includegraphics[scale=0.5]{figure/read_fault_multi.png}
        
\end{frame}

\begin{frame}{Porting initialization.f90 ke initialization.hs}
    $GET\_LANDSLIDE\_PARAMETERS$
    \begin{itemize}
        \item $GET\_LANDSLIDE\_PARAMETERS$ membaca semua parameter dari $landslide.ctl$.
        \item $GET\_LANDSLIDE\_PARAMETERS$ berisi $\pm$ 40 line untuk membaca $landslide.ctl$.
    \end{itemize}
    \includegraphics[scale=0.5]{figure/read_landslide.png}
\end{frame}

\end{document}